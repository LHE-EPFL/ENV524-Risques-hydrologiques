% -*- TeX:FR -*-
\documentclass[a4paper]{article}
% NE RIEN MODIFIER DANS LES ENTETES !!!
% DO NOT MODIFY THIS PART
\usepackage{lmodern}
\usepackage{bm}

\usepackage[T1]{fontenc}
\TeXXeTstate=1 %activer XeTTeX
\usepackage[utf8]{inputenc}
\usepackage[body={17cm,25cm},top=2.5cm,left=2.5cm]{geometry}



%\usepackage[numbers, sort&compress]{natbib}
\usepackage[dvips]{graphicx}


 
\usepackage{manfnt}
\usepackage{stmaryrd}
\usepackage{textcomp}
\usepackage{pifont}




\usepackage{amsmath}
\usepackage{amssymb}
\usepackage{graphics}
\usepackage{fancyhdr}
 \usepackage{wrapfig}
%\usepackage{frenchle} % utiliser de préférence frenchle http://www.efrench.org/doc/frenchle.pdf
\usepackage[frenchb]{babel} % sinon prendre l'option frenchb dans babel
% Utilisation du package caption pour changer le nom des titre des tableaux
\usepackage{caption}
\captionsetup[table]{name = Tableau}


\newcommand{\D}{\mathrm{d}}
\newcommand{\cube}{$^3$}



\begin{document}

\pagestyle{fancy}
\fancyhead{}
\chead{\textit{Risques hydrologiques et aménagement du territoire - 2021 -- \textbf{TD 4}} }



\huge
%\begin{center}
\textsc{\textbf{Séance 4 :  Loi de Pareto et extrapolations}}
%\end{center}


% \vspace{35pt}
\small
   \vspace{10pt}
    \textbf{Professeur :} Christophe \textsc{Ancey} 

    \vspace{10pt}
   \textbf{Assistants :} Clemente \textsc{Gotelli}, Mehrdad \textsc{Kiani} \\
    \vspace{10pt}
\hrule
\small

\section*{Ajustements et extrapolations}

Nous allons comparer ici la pluie centennale à Davos obtenue par extrapolation grâce aux quatre méthodes vues en cours, à savoir :\\

\begin{itemize}
 \item \textbf{Méthodes à seuils:} Méthode du renouvellement et loi généralisée de Pareto

 \item \textbf{Méthodes par blocs : }Méthodes des moments et du maximum de vraisemblance

\end{itemize}
\section{Méthode du renouvellement}
On rappelle que la méthode du renouvellement cherche à représenter les évènements extrêmes grâce à une loi de Poisson (comptabilisant le nombre d'évènements pour une période donnée) et une loi exponentielle (traduisant la distribution de ces évènements dans le temps). \\~\\ Pour une période de retour $T$ très grande, on obtient le quantile en fonction de la période de retour grâce à la formule suivante:\\
\begin{equation}\label{renouv}
 C=s+\frac{\mbox{ln}\lambda}{\mu}+\frac{1}{\mu}\mbox{ln}T+O(T^{-1})
\end{equation}
Avec les paramètres :
\begin{itemize}
 \item $\lambda=n_s/n_a$
 \item $\mu=1/(\tilde{C}-s)$
\end{itemize}
\begin{enumerate}
\item Tracer l'histogramme des précipitations à Davos couvrant $n_a$ années et déterminer à priori une valeur $s$ du seuil possible. Ne retenir que les $n_s$ précipitations dépassant le seuil par la suite.
\item Classer les valeurs de l'échantillon par ordre croissant et calculer $\tilde{C}$, sa moyenne empirique.
\item A chaque valeur de l'échantillon de rang $i$, on affecte la période de retour :
\begin{equation}
 T_i=\frac{n_a}{n_s}\frac{n_s+1}{i}
\end{equation}
\item Calculer les paramètres $\lambda$ et $\mu$, puis tracer dans un diagramme $(T,C)$ la variation du quantile $C$ en fonction de la période de retour $T$, grâce à la l'équation 1. Comparer avec les valeurs empiriques.
\item Quel est la valeur de la pluie centennale ? $C_{100}=$
\item Changer le seuil tel que $s_2>s$ puis $s_3<s$, et tracer les lois obtenues sur la même fenêtre. Quelle influence cela a t-il sur l'extrapolation? Comment choisir un « bon~» seuil selon vous?
\end{enumerate}
\section{Loi de Pareto}
\small \textit{~\\ Vilfredo Pareto, né le 15 juillet 1848 à Paris et mort le 19 août 1923 à Céligny (Suisse), est un sociologue et économiste italien. Il demeure célèbre pour son observation des 20 \% de la population qui possèdent 80 \% des richesses en Italie, généralisée plus tard en distribution de Pareto. Cette observation a été étendue à d'autres domaines sous le terme de « principe de Pareto ».  Il définit la notion d'optimum paretien comme une situation d'ensemble dans laquelle un individu ne peut améliorer sa situation sans détériorer celle d'un autre individu. \\
 \vspace{5pt}}

La loi de Pareto s'exprime ainsi :
\begin{equation}
 G(x)=1-{\left(1+\frac{\tilde{\xi}x}{\tilde{\sigma}}\right)}^{-1/\tilde{\xi}}
\end{equation}
On cherche à déterminer ses paramètres. On peut obtenir que pour tout seuil $s>s_0$, l'équation suivante est vérifiée :
\begin{equation}
	E[X-s|X>s] = \frac{\tilde{\sigma_s}}{1-\tilde{\xi}} = \frac{\tilde{\sigma_{s_0}} +
\tilde{\xi}(s-s_0)}{1-\tilde{\xi}}
\end{equation}
$\tilde{\xi}/(1-\tilde{\xi})$ est donc la pente et $(\tilde{\sigma_{s_0}} -
\tilde{\xi}s_0)/(1-\tilde{\xi})$ l'ordonnée à l'origine de la fonction $f(s)=E[X-s|X>s]$ \\
\begin{enumerate}
\item Tracer $E(X-s|X>s)$ en fonction de $s$ et déterminer a priori sur quel domaine le
paramètre $\tilde{\xi}$ pourra être ajusté. \textit{(calculer la moyenne de la distance des valeurs qui dépassent le seuil par rapport à ce même seuil)}
\item Pour un domaine fixé, calculer les paramètres de la loi de Pareto qui approchent le mieux cette distribution (vous pouvez utiliser la commande \texttt{polyfit}). Tracer la droite ajustée sur le même graphique que $f(s)$. A quelle type de loi de valeurs extrêmes peut-on comparer la loi de Pareto ?
\item Tracer le quantile $x_{p}(T)$ ainsi ajusté et comparer le aux données de l'échantillon.
\begin{equation}
\begin{split}
\mbox{si }\tilde{\xi}\ne 0, x_p(T) &=
s+ \frac{\tilde{\sigma}}{\tilde{\xi}}\left(\left(T\frac{n_s}{n_a}\right)^{\tilde{\xi}}-1\right) \\
  \mbox{si } \tilde{\xi} =0, x_p(T) &= s+\tilde{\sigma}\mbox{ln}\left(T\frac{n_s}{n_a}\right)
 \end{split}
\end{equation}
\item Ajuster maintenant les paramètres de la loi de Pareto grâce à la méthode d'optimisation de votre choix pour 3 seuils différents (max. de vraisemblance, Hasting-Metropolis, méthode des moments, etc) \textbf{[gpfit]}
\item Reproduire ces résultats sur un même graphique et les comparer.
\item Donner la valeur du quantile extrapolé $C_{100}=$
\end{enumerate}

\section{Maximum de vraisemblance}
\begin{enumerate}
 \item Calculer les maxima annuels.
 \item Ajuster une loi de valeurs extrêmes grâce au maximum de vraisemblance.
\item Tracer sur la même figure, le quantile obtenu $x_{ev}(T)$ via la méthode par bloc et les quantiles $x_{p}(T)$ et $C_s(T)$ obtenus via les méthodes à seuil. Conclure.
\end{enumerate}
\end{document}
